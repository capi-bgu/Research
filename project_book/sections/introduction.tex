\begin{document}
phone than with any other human. However, despite this "intimacy", the laptop has no idea how we are feeling. 
It has no idea if we are happy, mad, or just having a bad day. Today's technology has no emotional intelligence. 
So, what if our technology could sense our emotions? What if our devices could sense how we felt and reacted accordingly, 
just how an emotionally intelligent friend would?  Imagine yourself, for example, having a bad day, being through sad events,
and can't handle any sad news today. The computer can sense your sadness, show you less sad news and more happy ones, 
or simply play you a song it knows tends to make you happy.
These questions led us to research the field of Affective-Computing and explore the possibilities of applying machine learning to the problem of 
emotionally intelligent computers.

After an in-depth process of searching for applications that realize our vision, 
we realized that although the field does exist and is researched under the field of Affective-Computing, 
it is entirely in its infancy. Some companies and laboratories try to solve the problem with different approaches. 
There were attempts to identify people's emotions using facial expressions, analyze emotions using brain waves from an EEG machine, 
and process data from various physiological sensors such as temperature, blood pressure, heart rate, Etc.

These approaches are problematic for the everyday user - a person who sits in front of the computer or phone, chats, 
plays, works, or simply watches a movie. The approaches that use custom sensors require the user to purchase those sensors, 
which are very expensive and know how to operate them. These sensors are also very intrusive to the user's privacy. 
Technologies that are based only on facial expressions are also problematic. Most applications we found using this approach use heavy neural network
models trained on a group of people who sat in front of a camera and made expressions suitable for emotions - smile to show joy, raise eyebrows to show surprise.
As a result, the models do not identify the users' emotions but their facial expressions - which do not always match the emotion, moreover in different people, 
the same emotion can be represented by different expressions [image from Lisa Fe's book]. 
When sitting alone in front of a computer, most people do not emphasize their facial expressions in such a "drastic" way as the actors by whom the 
models were trained.

There are different approaches in the literature, such as learning from keyboard and mouse dynamics. 
A limited number of studies use these approaches. The studies we have found do not reveal the data, 
the data collection system and focus on a single channel and one type of emotion without comparison with different methods and approaches.

In our research, we tried to examine a proposal to streamline existing methods and their integration to maintain the users' 
privacy to the maximum extent and facilitate future research. We use several channels simultaneously to collect user data and extract 
usfull features from each channel:

\begin{itemize}
    \item Keyboard - we extract features like typing speed, and average time between key presses.
    \item Mouse - we extract features such as movment speed, number of clicks, and press duration.
    \item Camera - we focus the image on the user’s face to remove noise.  
\end{itemize}

These channels are accessible to almost any user without unique components while being simple to collect data from without privacy intrusion. 
Furthermore, we take the approach of personal models for each user because, as we said before, different people can represent the same emotion 
with different facial expressions and different usage patterns.

During the study, we built a system for collecting data from users. 
Then, using the data we collected, we built models for each user for each channel we mentioned above and models that combine the channels. 
From the results we got, we saw that the face models brought the best results. The combined models yielded results similar to the face models,
while the keyboard and mouse were not as good, though better than expected.

\end{document}