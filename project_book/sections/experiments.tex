\begin{document}
   
We can divide our experiments across three categories, facial expression recognition,
keyboard and mouse dynamics, and ensemble. We will also discuss our data gathering techniques here.
\par

We have two types of data gathering. Mood induced data gathering will be performed in a controlled environment. Each participant
will be exposed to a series of videos, movies, games, or some other emotion-inducing media depending on the participant's preferences.
The method of mood induction could be different between participants.
While watching the videos, we will record the participants and label their emotions on the VAD scale and the corresponding
categorical emotions scale. During breaks in the experiment, we will ask the participants to perform some computer tasks and record
their keyboard and mouse. The participants will be recorded during the task, as well. We hope to have at least 10 participants, each recorded
between 1 and 3 hours in this experiment.

A second way we will collect data is using an "out in the wild" experiment. We will let a few people (including ourselves)
install the data gathering system we built and use it for between 1 and 3 weeks. The participants will be occasionally prompted
to label their emotions on the VAD scale. We will collect clips of the participants a few seconds before they are prompted to
label their emotions, as well as the last few seconds, or minutes of keyboard and mouse usage,
and some general statistics of keyboard and mouse use in accordance to what was described in \ref{section:keyboard_mouse}.
\par

For the keyboard and mouse, we have three dimensions along which we need to experiment: model type, personal vs. general model,
classification vs. regression model. We will only use the data we collected for this experiment because we have not found any publicly
available data for keyboard and mouse dynamics with emotion labels. 
\par

The facial expression recognition experiments will be more structured as the model search space is practically infinite.
We will experiment with the three model types we discussed in \ref{section:fer}. Also, experimenting with regression variations of said
models and with personalized variations. We will first focus on the regression vs. classification and model architecture, as most data sets
do not differentiate the participants. Once our data gathering is complete, we will experiment more with making the models personalized.
\par

The ensemble experiments will have to come last in our timeline. Here we will experiment with combining the facial expression models and the
keyboard and mouse models. We will only select some of the best models from each category and use different ensemble methods to combine them.
\par

The most significant uncertainty in our research comes from the data gathering process. Collecting labels that are representative of emotions
is extremely difficult, if not impossible. We hope that our personalized models, regression models, and time series analysis can better use this
labeling than previous methods. Our biggest challenge will be selecting meaningful features, whether it is by modifying our neural network models
or analyzing the keyboard and mouse features. 
\end{document}
