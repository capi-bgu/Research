\documentclass[../main.tex]{subfiles}

\begin{document}
As mentioned in the Emotional Psychology paragraph, emotion plays a massive role in our everyday life.
Nowadays, people share most of the time with their machines. Although people express their emotions 
even when they are alone in front of the computer, and with the fact that machines have a lot of 
IQ (Intelligence quotient), they still can’t process those emotions because they have no EQ (Emotional quotient). 
There is no way to pass users’ affective state to or through the machine.
\par

To solve that problem, in the year 1995, Rosalind W. Picard coined the phrase “Affective Computing” 
in her book\cite{affective_computing_book}. She defined affective computing as “computing that relates to, 
arises from or deliberately influences emotions.” In other words, giving machines an emotional ability- 
how to make the machines understand emotions, react to them, and express them. 
Since R.W. Picard coined the term, she founded the MIT affective computing lab, which did a lot of
research in the field. For example, one of the more exciting
projects in the field is “AutoEmotive” \cite{AutoEmotive}. In this project, they collect 
some features like pressure on the steering wheel, facial expressions from the camera, 
and voice tone. Then detects how much stress the driver feels and reacts accordingly. 
For instance, when the system recognizes the driver is tired, it will suggest lowering the air
conditioner temperature or replace the music. When the system recognizes the driver is stressed, it will
suggest calm music.
\end{document}

