\begin{titlepage}
    \begin{center}
        \vspace*{1cm}
        
        \includegraphics[width=0.1\textwidth]{figures/bgu.png}\\
        \selectlanguage{hebrew}
        אוניברסיטת בן-גוריון בנגב\\
        הפקולטה להנדסה\\
        המחלקה למערכות תוכנה ומידע
        \selectlanguage{english}
        \vspace{2cm}
        
        \selectlanguage{hebrew}
        {\Large IPAC - חיזוי רגשות}
        \selectlanguage{english}
        
        \vspace{1.5cm}
  
        \selectlanguage{hebrew}
        הוד טויטו, שהם צרפתי, יובל חורמ-יאן, רון זיידמן
        \selectlanguage{english}
        \vspace{1cm}
        
        \selectlanguage{hebrew}
        בהנחיית פרופסור יובל שחר
        \selectlanguage{english}
  
        \vspace{1cm}
  
  
        \selectlanguage{hebrew}
        \begin{abstract}
          בפרויקט זה, חקרנו את האפשרויות של חיזוי רגשות המשתמש על בסיס שיטות שונות של למידת מכונה. במהלך המחקר שלנו הסתכלנו על מחקרים קודמים שהציעו גישות שונות לפתרון בעיה זו, למשל על ידי הסתכלות על מידע מהמקלדת, מהעכבר, מהמצלמה וכן גישות יותר חודרניות כגון לחץ דם, דופק או גלים חשמליים במוח. 
  על מנת להשיג את המידע הדרוש, יצרנו שתי חבילות תוכנה: יצרנו ספרייה שאוספת את המידע בערוצים שונים, כאשר ערוצי ברירת המחדל הנתונים בספרייה הם המקלדת, העכבר והמצלמה. נוסף על כך, יצרנו ממשק משתמש שמבוסס על הספרייה שיצרנו על מנת לאסוף את המידע מהמשתמש ולתייג את אותו המידע ברגש. נתנו את תוכנת הממשק לקבוצה של נסיינים על מנת לאסוף מהם מידע בערוצים השונים. לאחר מכן, ניתחנו את המידע שאספנו ובנינו מספר מודלים שיכולים לחזות את רגשות המשתמש, וכן את רמת חיוביותו של המשתמש ברמות דיוק שונות. כמו שציפינו, המודלים אשר הסתמכו על המידע מהבעות הפנים הניבו את התוצאות הטובות ביותר, ומודלים אשר הסתמכו על המידע מהמקלדת והעכבר השיגו תוצאות פחותות מכך אך עדיין אלו תוצאות דומות למחקרים הקודמים שסקרנו אשר ניסו דרכים דומות, ואף מתעלות על חלק מהם. כמו כן, לאחר סקירה של הגדרת הרגשות, ניסינו לחזות גם מימדים שונים ומתמשכים של רגשות חוץ מרגשות קטגוריים. מימדים אלה מתארים תמונה יותר מפורטת של רגש המשתמש, שגם פותחת אפיק להמשך מחקר בעניין. במהלך יצירת המודלים, אימנו אותם על ייצוגים שונים של הרגש, השתמשנו במודלים המורכבים מכמה מודלים, וגם ניסינו שימוש של ערוצים שונים ביחד.
  
        \end{abstract} 
        \selectlanguage{english}
        
        
        \vfill
        
        \selectlanguage{hebrew}
        יוני 1202
        \selectlanguage{english}
    \end{center}
  \end{titlepage}