\documentclass[../main.tex]{subfiles}

\begin{document}
Machine Learning is a subfield in Computer Science and in Artificial Intelligence. 
The field deals with developing and using algorithms designed to allow the machines to 
learn from examples and operate in various computational tasks that classical programming can not. 
In other words, the machine is searching for patterns and trying to generalize them to 
the whole given problem. With this generalization, it can infer knowledge about new data.
\par
 
In Machine Learning, there are several common problems, and we are focusing on classification 
and regression.
In classification problems, the machine receives a set of data and labels for each data point and 
tries to learn a mapping from given data to the given labels. 
There are many classification algorithms \cite{machine_learning_classification_algorithms}. For example:
\begin{itemize}
    \item SVM will try to find a hyperplane that will split the data into given two classes as precisely as possible. 
    \item Decision Trees will try to split the data according to its features by maximizing information gain.
    \item KNN will try to look at the nearest K instances in the data and will choose the right label according to them.
    \item ANN will try to fix his network, built from input layer, hidden layer, and output layer, to fit the given dataset. 
\end{itemize}
On the other hand, in regression problems, the machine receives a set of data and continuous values for 
each data point. The machine tries to learn a function that will transform given features into a 
numeric value. That function needs to fit its weights to the given dataset and its values. 
Some of the classification algorithms mentioned above, such as SVM, ANN, Decision Trees, can be 
modified to fit regression problems as well.

\end{document}