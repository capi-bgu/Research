\documentclass{article}


\usepackage{arxiv}

\usepackage[utf8]{inputenc} % allow utf-8 input
\usepackage[T1]{fontenc}    % use 8-bit T1 fonts
\usepackage{hyperref}       % hyperlinks
\usepackage{enumerate}      % pretty enumration
\usepackage{url}            % simple URL typesetting
\usepackage{booktabs}       % professional-quality tables
\usepackage{amsfonts}       % blackboard math symbols
\usepackage{nicefrac}       % compact symbols for 1/2, etc.
\usepackage{microtype}      % microtypography
\usepackage{graphicx}
\graphicspath{ {./images/} }
\usepackage{subfiles}

\title{CAPI \\ Literature Review}


\author{
  Shoham Zarfti \\
  Department of Software and Information Systems \\
  Ben-Gurion University \\
   \And
 Yuval Khoramian \\
  Department of Software and Information Systems\\
  Ben-Gurion University \\
  \And
 Hod Twito \\
  Department of Software and Information Systems\\
  Ben-Gurion University \\
  \And
 Ron Ziedman \\
  Department of Software and Information Systems\\
  Ben-Gurion University \\
  %% Coauthor \\
  %% Affiliation \\
  %% Address \\
  %% \texttt{email} \\
  %% \And
  %% Coauthor \\
  %% Affiliation \\
  %% Address \\
  %% \texttt{email} \\
  %% \And
  %% Coauthor \\
  %% Affiliation \\
  %% Address \\
  %% \texttt{email} \\
}

\begin{document}
\maketitle
\begin{abstract}
Fill in executive summary when we are done with the reset of the review.
\end{abstract}


 keywords can be removed
\keywords{First keyword \and Second keyword \and More}

\section{Project Background}
Our project exists on the intersections between psychology and machine learning, as such we explore in this section three fields, psychology, classical machine learning, and computer vision.

\subsection{Psychology}
\subfile{sections/psychology}

\subsection{Affective Computing}
\subfile{sections/affective_computing}

\subsection{Classical Machine Learning}
\subfile{sections/classical_ml}


\subsection{Computer Vision}
\subfile{sections/computer_vision}


\section{Related Work}
In this section we dive deeper to into similar work done in this field.

\subsection{Facial Expression Recognition}
\subfile{sections/facial_expression_recognition}

\subsection{Emotion Recognition Using Keyboard and Mouse}
\subfile{sections/emotion_recognition_keyboard_mouse}

Until now, we have discussed predicting categorical emotions because most of the research in the 
field focuses on categorical emotions. As mentioned in the psychology background section, 
we can also represent emotion as continuous values in the dimensional model. 
One of our goals is to research the dimensional model using the approaches we have stated before. 
We will try to predict a numeric value for each of the affective dimensions using 
regression methods.
\par

Buechel et al. \cite{emotion_regression} suggest a similar approach for NLP sentiment analysis. 
Their paper describes a regression model that predicts values in the VAD emotional 
space \cite{VAD_model} from text data and then transforming the results to Ekman's BE model \cite{Ekman_Theory}. 
Although we are not using textual data, we feel the accomplishments achieved in their 
paper are encouraging. The authors claimed their VAD-based system achieves state-of-the-art 
performance in three out of six emotion categories.
\par

In conclusion, we believe we can contribute to this field by improving and combining some of the 
methods we have discussed. Our core idea is to create an ensemble model that will combine the 
facial expression model with the keyboard and mouse analysis model. 
Both models will build on the research we have reviewed, combining personalized models with 
regression methods for predicting VAD-space values and time series methods.


\section{Research Plan}
\subfile{sections/research_plan}



\bibliographystyle{abbrv}
%\bibliographystyle{apalike}  
\bibliography{references.bib}

\end{document}
