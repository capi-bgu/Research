\documentclass{article}


\usepackage{arxiv}

\usepackage[utf8]{inputenc} % allow utf-8 input
\usepackage[T1]{fontenc}    % use 8-bit T1 fonts
\usepackage{hyperref}       % hyperlinks
\usepackage{enumerate}      % pretty enumeration
\usepackage{url}            % simple URL typesetting
\usepackage{booktabs}       % professional-quality tables
\usepackage{amsfonts}       % blackboard math symbols
\usepackage{nicefrac}       % compact symbols for 1/2, etc.
\usepackage{svg}
\usepackage{microtype}      % microtypography
\usepackage{graphicx}
\graphicspath{ {./images/} }
\usepackage{subfiles}

\title{CAPI \\ Literature Review}


\author{
  Shoham Zarfti \\
  Department of Software and Information Systems \\
  Ben-Gurion University \\
   \And
 Yuval Khoramian \\
  Department of Software and Information Systems\\
  Ben-Gurion University \\
  \And
 Hod Twito \\
  Department of Software and Information Systems\\
  Ben-Gurion University \\
  \And
 Ron Ziedman \\
  Department of Software and Information Systems\\
  Ben-Gurion University \\
  %% Coauthor \\
  %% Affiliation \\
  %% Address \\
  %% \texttt{email} \\
  %% \And
  %% Coauthor \\
  %% Affiliation \\
  %% Address \\
  %% \texttt{email} \\
  %% \And
  %% Coauthor \\
  %% Affiliation \\
  %% Address \\
  %% \texttt{email} \\
}

\begin{document}
\maketitle
\begin{abstract}
  In this project, we will attempt to contribute to the field of affective computing by applying some new techniques to the problem of inferring users'
  emotions. Our goal is to create a model that combines elements from different affective computing subfields.
  We will use facial expression recognition and keyboard analysis. We will create personalized models to infer a user's emotional state online
  and offline. These models will use classification algorithms as stated in the related work as well as our new approach of regression. In other words,
  the question we would like to answer is, can some ensemble of the methods above improve on existing methods. We feel that significant
  developments in this space could allow computer systems to better interact with users in various ways, from filtering out bad news if the user
  is feeling down to helping psychologists track their patients' mental health.
\end{abstract}


\section{Project Background}
Our project exists on the intersections between psychology and machine learning.
This section explores four fields, machine learning, computer vision, psychology, and affective computing.
We felt these fields could give us a rich foundation to build our project upon.


\subsection{Psychology}
\subfile{sections/psychology}

\subsection{Affective Computing}
\subfile{sections/affective_computing}

\subsection{Machine Learning}
\subfile{sections/machine_learning}

\subsection{Computer Vision}  \label{section:cv}
\subfile{sections/computer_vision}


\section{Related Work}
How to infer emotion is a question that was researched for a long time with several approaches and fields.
Some researchers from emotional psychology tried to investigate how to infer emotions in several ways, and they all agree
it is a very difficult task. There are many approaches and technologies to infer emotions.
For instance, using a microphone to get a user’s voice, a camera to take a photo of a user’s face, computer input (keyboard, mouse) to record a user’s
input behavior or special equipment to measure a user’s physiological state (blood pressure, sweat).
Although the problem is mostly solved using some of the previews mentioned approaches, most of them can not be used in the "real world".
Most of the mainly used approaches like physiological state require special devices to measure that the "everyday user" does not have.
Moreover, many people will claim that those approaches are very intrusive and will object to using them daily.
To deal with this, we will focus on inferring emotions through camera and computer input, which are devices every user has,
and we will attempt to make their use as private to the user as possible.

\subsection{Facial Expression Recognition} \label{section:fer}
\subfile{sections/facial_expression_recognition}

\subsection{Emotion Recognition Using Keyboard and Mouse} \label{section:keyboard_mouse}
\subfile{sections/emotion_recognition_keyboard_mouse}

Until now, we have discussed predicting categorical emotions because most of the research in the 
field focuses on categorical emotions. As mentioned in the psychology background section, 
we can also represent emotion as continuous values in the dimensional model. 
One of our goals is to research the dimensional model using the approaches we have stated before. 
We will try to predict a numeric value for each of the affective dimensions using 
regression methods.
\par

Buechel et al. \cite{emotion_regression} suggest a similar approach for NLP sentiment analysis. 
Their paper describes a regression model that predicts values in the VAD emotional 
space \cite{VAD_model} from text data and then transforming the results to Ekman's BE model \cite{Ekman_Theory}. 
Although we are not using textual data, we feel the accomplishments achieved in their 
paper are encouraging. The authors claimed their VAD-based system achieves state-of-the-art 
performance in three out of six emotion categories.
\par

In conclusion, we believe we can contribute to this field by improving and combining some of the 
methods we have discussed. Our core idea is to create an ensemble model that will combine the 
facial expression model with the keyboard and mouse analysis model. 
Both models will build on the research we have reviewed, combining personalized models with 
regression methods for predicting VAD-space values and time series methods.


\section{Research Plan}
\subfile{sections/research_plan}

\bibliographystyle{abbrv}
%\bibliographystyle{apalike}  
\bibliography{references.bib}

\end{document}
